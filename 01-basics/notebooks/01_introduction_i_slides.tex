
% Default to the notebook output style

    


% Inherit from the specified cell style.




    
\documentclass{article}

    
    
    \usepackage{graphicx} % Used to insert images
    \usepackage{adjustbox} % Used to constrain images to a maximum size 
    \usepackage{color} % Allow colors to be defined
    \usepackage{enumerate} % Needed for markdown enumerations to work
    \usepackage{geometry} % Used to adjust the document margins
    \usepackage{amsmath} % Equations
    \usepackage{amssymb} % Equations
    \usepackage{eurosym} % defines \euro
    \usepackage[mathletters]{ucs} % Extended unicode (utf-8) support
    \usepackage[utf8x]{inputenc} % Allow utf-8 characters in the tex document
    \usepackage{fancyvrb} % verbatim replacement that allows latex
    \usepackage{grffile} % extends the file name processing of package graphics 
                         % to support a larger range 
    % The hyperref package gives us a pdf with properly built
    % internal navigation ('pdf bookmarks' for the table of contents,
    % internal cross-reference links, web links for URLs, etc.)
    \usepackage{hyperref}
    \usepackage{longtable} % longtable support required by pandoc >1.10
    \usepackage{booktabs}  % table support for pandoc > 1.12.2
    \usepackage{ulem} % ulem is needed to support strikethroughs (\sout)
    

    
    
    \definecolor{orange}{cmyk}{0,0.4,0.8,0.2}
    \definecolor{darkorange}{rgb}{.71,0.21,0.01}
    \definecolor{darkgreen}{rgb}{.12,.54,.11}
    \definecolor{myteal}{rgb}{.26, .44, .56}
    \definecolor{gray}{gray}{0.45}
    \definecolor{lightgray}{gray}{.95}
    \definecolor{mediumgray}{gray}{.8}
    \definecolor{inputbackground}{rgb}{.95, .95, .85}
    \definecolor{outputbackground}{rgb}{.95, .95, .95}
    \definecolor{traceback}{rgb}{1, .95, .95}
    % ansi colors
    \definecolor{red}{rgb}{.6,0,0}
    \definecolor{green}{rgb}{0,.65,0}
    \definecolor{brown}{rgb}{0.6,0.6,0}
    \definecolor{blue}{rgb}{0,.145,.698}
    \definecolor{purple}{rgb}{.698,.145,.698}
    \definecolor{cyan}{rgb}{0,.698,.698}
    \definecolor{lightgray}{gray}{0.5}
    
    % bright ansi colors
    \definecolor{darkgray}{gray}{0.25}
    \definecolor{lightred}{rgb}{1.0,0.39,0.28}
    \definecolor{lightgreen}{rgb}{0.48,0.99,0.0}
    \definecolor{lightblue}{rgb}{0.53,0.81,0.92}
    \definecolor{lightpurple}{rgb}{0.87,0.63,0.87}
    \definecolor{lightcyan}{rgb}{0.5,1.0,0.83}
    
    % commands and environments needed by pandoc snippets
    % extracted from the output of `pandoc -s`
    \providecommand{\tightlist}{%
      \setlength{\itemsep}{0pt}\setlength{\parskip}{0pt}}
    \DefineVerbatimEnvironment{Highlighting}{Verbatim}{commandchars=\\\{\}}
    % Add ',fontsize=\small' for more characters per line
    \newenvironment{Shaded}{}{}
    \newcommand{\KeywordTok}[1]{\textcolor[rgb]{0.00,0.44,0.13}{\textbf{{#1}}}}
    \newcommand{\DataTypeTok}[1]{\textcolor[rgb]{0.56,0.13,0.00}{{#1}}}
    \newcommand{\DecValTok}[1]{\textcolor[rgb]{0.25,0.63,0.44}{{#1}}}
    \newcommand{\BaseNTok}[1]{\textcolor[rgb]{0.25,0.63,0.44}{{#1}}}
    \newcommand{\FloatTok}[1]{\textcolor[rgb]{0.25,0.63,0.44}{{#1}}}
    \newcommand{\CharTok}[1]{\textcolor[rgb]{0.25,0.44,0.63}{{#1}}}
    \newcommand{\StringTok}[1]{\textcolor[rgb]{0.25,0.44,0.63}{{#1}}}
    \newcommand{\CommentTok}[1]{\textcolor[rgb]{0.38,0.63,0.69}{\textit{{#1}}}}
    \newcommand{\OtherTok}[1]{\textcolor[rgb]{0.00,0.44,0.13}{{#1}}}
    \newcommand{\AlertTok}[1]{\textcolor[rgb]{1.00,0.00,0.00}{\textbf{{#1}}}}
    \newcommand{\FunctionTok}[1]{\textcolor[rgb]{0.02,0.16,0.49}{{#1}}}
    \newcommand{\RegionMarkerTok}[1]{{#1}}
    \newcommand{\ErrorTok}[1]{\textcolor[rgb]{1.00,0.00,0.00}{\textbf{{#1}}}}
    \newcommand{\NormalTok}[1]{{#1}}
    
    % Additional commands for more recent versions of Pandoc
    \newcommand{\ConstantTok}[1]{\textcolor[rgb]{0.53,0.00,0.00}{{#1}}}
    \newcommand{\SpecialCharTok}[1]{\textcolor[rgb]{0.25,0.44,0.63}{{#1}}}
    \newcommand{\VerbatimStringTok}[1]{\textcolor[rgb]{0.25,0.44,0.63}{{#1}}}
    \newcommand{\SpecialStringTok}[1]{\textcolor[rgb]{0.73,0.40,0.53}{{#1}}}
    \newcommand{\ImportTok}[1]{{#1}}
    \newcommand{\DocumentationTok}[1]{\textcolor[rgb]{0.73,0.13,0.13}{\textit{{#1}}}}
    \newcommand{\AnnotationTok}[1]{\textcolor[rgb]{0.38,0.63,0.69}{\textbf{\textit{{#1}}}}}
    \newcommand{\CommentVarTok}[1]{\textcolor[rgb]{0.38,0.63,0.69}{\textbf{\textit{{#1}}}}}
    \newcommand{\VariableTok}[1]{\textcolor[rgb]{0.10,0.09,0.49}{{#1}}}
    \newcommand{\ControlFlowTok}[1]{\textcolor[rgb]{0.00,0.44,0.13}{\textbf{{#1}}}}
    \newcommand{\OperatorTok}[1]{\textcolor[rgb]{0.40,0.40,0.40}{{#1}}}
    \newcommand{\BuiltInTok}[1]{{#1}}
    \newcommand{\ExtensionTok}[1]{{#1}}
    \newcommand{\PreprocessorTok}[1]{\textcolor[rgb]{0.74,0.48,0.00}{{#1}}}
    \newcommand{\AttributeTok}[1]{\textcolor[rgb]{0.49,0.56,0.16}{{#1}}}
    \newcommand{\InformationTok}[1]{\textcolor[rgb]{0.38,0.63,0.69}{\textbf{\textit{{#1}}}}}
    \newcommand{\WarningTok}[1]{\textcolor[rgb]{0.38,0.63,0.69}{\textbf{\textit{{#1}}}}}
    
    
    % Define a nice break command that doesn't care if a line doesn't already
    % exist.
    \def\br{\hspace*{\fill} \\* }
    % Math Jax compatability definitions
    \def\gt{>}
    \def\lt{<}
    % Document parameters
    \title{Slides: Introduction I }
    \author{Florencia Noriega}
    
    
    

    % Pygments definitions
    
\makeatletter
\def\PY@reset{\let\PY@it=\relax \let\PY@bf=\relax%
    \let\PY@ul=\relax \let\PY@tc=\relax%
    \let\PY@bc=\relax \let\PY@ff=\relax}
\def\PY@tok#1{\csname PY@tok@#1\endcsname}
\def\PY@toks#1+{\ifx\relax#1\empty\else%
    \PY@tok{#1}\expandafter\PY@toks\fi}
\def\PY@do#1{\PY@bc{\PY@tc{\PY@ul{%
    \PY@it{\PY@bf{\PY@ff{#1}}}}}}}
\def\PY#1#2{\PY@reset\PY@toks#1+\relax+\PY@do{#2}}

\expandafter\def\csname PY@tok@gd\endcsname{\def\PY@tc##1{\textcolor[rgb]{0.63,0.00,0.00}{##1}}}
\expandafter\def\csname PY@tok@gu\endcsname{\let\PY@bf=\textbf\def\PY@tc##1{\textcolor[rgb]{0.50,0.00,0.50}{##1}}}
\expandafter\def\csname PY@tok@gt\endcsname{\def\PY@tc##1{\textcolor[rgb]{0.00,0.27,0.87}{##1}}}
\expandafter\def\csname PY@tok@gs\endcsname{\let\PY@bf=\textbf}
\expandafter\def\csname PY@tok@gr\endcsname{\def\PY@tc##1{\textcolor[rgb]{1.00,0.00,0.00}{##1}}}
\expandafter\def\csname PY@tok@cm\endcsname{\let\PY@it=\textit\def\PY@tc##1{\textcolor[rgb]{0.25,0.50,0.50}{##1}}}
\expandafter\def\csname PY@tok@vg\endcsname{\def\PY@tc##1{\textcolor[rgb]{0.10,0.09,0.49}{##1}}}
\expandafter\def\csname PY@tok@m\endcsname{\def\PY@tc##1{\textcolor[rgb]{0.40,0.40,0.40}{##1}}}
\expandafter\def\csname PY@tok@mh\endcsname{\def\PY@tc##1{\textcolor[rgb]{0.40,0.40,0.40}{##1}}}
\expandafter\def\csname PY@tok@go\endcsname{\def\PY@tc##1{\textcolor[rgb]{0.53,0.53,0.53}{##1}}}
\expandafter\def\csname PY@tok@ge\endcsname{\let\PY@it=\textit}
\expandafter\def\csname PY@tok@vc\endcsname{\def\PY@tc##1{\textcolor[rgb]{0.10,0.09,0.49}{##1}}}
\expandafter\def\csname PY@tok@il\endcsname{\def\PY@tc##1{\textcolor[rgb]{0.40,0.40,0.40}{##1}}}
\expandafter\def\csname PY@tok@cs\endcsname{\let\PY@it=\textit\def\PY@tc##1{\textcolor[rgb]{0.25,0.50,0.50}{##1}}}
\expandafter\def\csname PY@tok@cp\endcsname{\def\PY@tc##1{\textcolor[rgb]{0.74,0.48,0.00}{##1}}}
\expandafter\def\csname PY@tok@gi\endcsname{\def\PY@tc##1{\textcolor[rgb]{0.00,0.63,0.00}{##1}}}
\expandafter\def\csname PY@tok@gh\endcsname{\let\PY@bf=\textbf\def\PY@tc##1{\textcolor[rgb]{0.00,0.00,0.50}{##1}}}
\expandafter\def\csname PY@tok@ni\endcsname{\let\PY@bf=\textbf\def\PY@tc##1{\textcolor[rgb]{0.60,0.60,0.60}{##1}}}
\expandafter\def\csname PY@tok@nl\endcsname{\def\PY@tc##1{\textcolor[rgb]{0.63,0.63,0.00}{##1}}}
\expandafter\def\csname PY@tok@nn\endcsname{\let\PY@bf=\textbf\def\PY@tc##1{\textcolor[rgb]{0.00,0.00,1.00}{##1}}}
\expandafter\def\csname PY@tok@no\endcsname{\def\PY@tc##1{\textcolor[rgb]{0.53,0.00,0.00}{##1}}}
\expandafter\def\csname PY@tok@na\endcsname{\def\PY@tc##1{\textcolor[rgb]{0.49,0.56,0.16}{##1}}}
\expandafter\def\csname PY@tok@nb\endcsname{\def\PY@tc##1{\textcolor[rgb]{0.00,0.50,0.00}{##1}}}
\expandafter\def\csname PY@tok@nc\endcsname{\let\PY@bf=\textbf\def\PY@tc##1{\textcolor[rgb]{0.00,0.00,1.00}{##1}}}
\expandafter\def\csname PY@tok@nd\endcsname{\def\PY@tc##1{\textcolor[rgb]{0.67,0.13,1.00}{##1}}}
\expandafter\def\csname PY@tok@ne\endcsname{\let\PY@bf=\textbf\def\PY@tc##1{\textcolor[rgb]{0.82,0.25,0.23}{##1}}}
\expandafter\def\csname PY@tok@nf\endcsname{\def\PY@tc##1{\textcolor[rgb]{0.00,0.00,1.00}{##1}}}
\expandafter\def\csname PY@tok@si\endcsname{\let\PY@bf=\textbf\def\PY@tc##1{\textcolor[rgb]{0.73,0.40,0.53}{##1}}}
\expandafter\def\csname PY@tok@s2\endcsname{\def\PY@tc##1{\textcolor[rgb]{0.73,0.13,0.13}{##1}}}
\expandafter\def\csname PY@tok@vi\endcsname{\def\PY@tc##1{\textcolor[rgb]{0.10,0.09,0.49}{##1}}}
\expandafter\def\csname PY@tok@nt\endcsname{\let\PY@bf=\textbf\def\PY@tc##1{\textcolor[rgb]{0.00,0.50,0.00}{##1}}}
\expandafter\def\csname PY@tok@nv\endcsname{\def\PY@tc##1{\textcolor[rgb]{0.10,0.09,0.49}{##1}}}
\expandafter\def\csname PY@tok@s1\endcsname{\def\PY@tc##1{\textcolor[rgb]{0.73,0.13,0.13}{##1}}}
\expandafter\def\csname PY@tok@kd\endcsname{\let\PY@bf=\textbf\def\PY@tc##1{\textcolor[rgb]{0.00,0.50,0.00}{##1}}}
\expandafter\def\csname PY@tok@sh\endcsname{\def\PY@tc##1{\textcolor[rgb]{0.73,0.13,0.13}{##1}}}
\expandafter\def\csname PY@tok@sc\endcsname{\def\PY@tc##1{\textcolor[rgb]{0.73,0.13,0.13}{##1}}}
\expandafter\def\csname PY@tok@sx\endcsname{\def\PY@tc##1{\textcolor[rgb]{0.00,0.50,0.00}{##1}}}
\expandafter\def\csname PY@tok@bp\endcsname{\def\PY@tc##1{\textcolor[rgb]{0.00,0.50,0.00}{##1}}}
\expandafter\def\csname PY@tok@c1\endcsname{\let\PY@it=\textit\def\PY@tc##1{\textcolor[rgb]{0.25,0.50,0.50}{##1}}}
\expandafter\def\csname PY@tok@kc\endcsname{\let\PY@bf=\textbf\def\PY@tc##1{\textcolor[rgb]{0.00,0.50,0.00}{##1}}}
\expandafter\def\csname PY@tok@c\endcsname{\let\PY@it=\textit\def\PY@tc##1{\textcolor[rgb]{0.25,0.50,0.50}{##1}}}
\expandafter\def\csname PY@tok@mf\endcsname{\def\PY@tc##1{\textcolor[rgb]{0.40,0.40,0.40}{##1}}}
\expandafter\def\csname PY@tok@err\endcsname{\def\PY@bc##1{\setlength{\fboxsep}{0pt}\fcolorbox[rgb]{1.00,0.00,0.00}{1,1,1}{\strut ##1}}}
\expandafter\def\csname PY@tok@mb\endcsname{\def\PY@tc##1{\textcolor[rgb]{0.40,0.40,0.40}{##1}}}
\expandafter\def\csname PY@tok@ss\endcsname{\def\PY@tc##1{\textcolor[rgb]{0.10,0.09,0.49}{##1}}}
\expandafter\def\csname PY@tok@sr\endcsname{\def\PY@tc##1{\textcolor[rgb]{0.73,0.40,0.53}{##1}}}
\expandafter\def\csname PY@tok@mo\endcsname{\def\PY@tc##1{\textcolor[rgb]{0.40,0.40,0.40}{##1}}}
\expandafter\def\csname PY@tok@kn\endcsname{\let\PY@bf=\textbf\def\PY@tc##1{\textcolor[rgb]{0.00,0.50,0.00}{##1}}}
\expandafter\def\csname PY@tok@mi\endcsname{\def\PY@tc##1{\textcolor[rgb]{0.40,0.40,0.40}{##1}}}
\expandafter\def\csname PY@tok@gp\endcsname{\let\PY@bf=\textbf\def\PY@tc##1{\textcolor[rgb]{0.00,0.00,0.50}{##1}}}
\expandafter\def\csname PY@tok@o\endcsname{\def\PY@tc##1{\textcolor[rgb]{0.40,0.40,0.40}{##1}}}
\expandafter\def\csname PY@tok@kr\endcsname{\let\PY@bf=\textbf\def\PY@tc##1{\textcolor[rgb]{0.00,0.50,0.00}{##1}}}
\expandafter\def\csname PY@tok@s\endcsname{\def\PY@tc##1{\textcolor[rgb]{0.73,0.13,0.13}{##1}}}
\expandafter\def\csname PY@tok@kp\endcsname{\def\PY@tc##1{\textcolor[rgb]{0.00,0.50,0.00}{##1}}}
\expandafter\def\csname PY@tok@w\endcsname{\def\PY@tc##1{\textcolor[rgb]{0.73,0.73,0.73}{##1}}}
\expandafter\def\csname PY@tok@kt\endcsname{\def\PY@tc##1{\textcolor[rgb]{0.69,0.00,0.25}{##1}}}
\expandafter\def\csname PY@tok@ow\endcsname{\let\PY@bf=\textbf\def\PY@tc##1{\textcolor[rgb]{0.67,0.13,1.00}{##1}}}
\expandafter\def\csname PY@tok@sb\endcsname{\def\PY@tc##1{\textcolor[rgb]{0.73,0.13,0.13}{##1}}}
\expandafter\def\csname PY@tok@k\endcsname{\let\PY@bf=\textbf\def\PY@tc##1{\textcolor[rgb]{0.00,0.50,0.00}{##1}}}
\expandafter\def\csname PY@tok@se\endcsname{\let\PY@bf=\textbf\def\PY@tc##1{\textcolor[rgb]{0.73,0.40,0.13}{##1}}}
\expandafter\def\csname PY@tok@sd\endcsname{\let\PY@it=\textit\def\PY@tc##1{\textcolor[rgb]{0.73,0.13,0.13}{##1}}}

\def\PYZbs{\char`\\}
\def\PYZus{\char`\_}
\def\PYZob{\char`\{}
\def\PYZcb{\char`\}}
\def\PYZca{\char`\^}
\def\PYZam{\char`\&}
\def\PYZlt{\char`\<}
\def\PYZgt{\char`\>}
\def\PYZsh{\char`\#}
\def\PYZpc{\char`\%}
\def\PYZdl{\char`\$}
\def\PYZhy{\char`\-}
\def\PYZsq{\char`\'}
\def\PYZdq{\char`\"}
\def\PYZti{\char`\~}
% for compatibility with earlier versions
\def\PYZat{@}
\def\PYZlb{[}
\def\PYZrb{]}
\makeatother


    % Exact colors from NB
    \definecolor{incolor}{rgb}{0.0, 0.0, 0.5}
    \definecolor{outcolor}{rgb}{0.545, 0.0, 0.0}



    
    % Prevent overflowing lines due to hard-to-break entities
    \sloppy 
    % Setup hyperref package
    \hypersetup{
      breaklinks=true,  % so long urls are correctly broken across lines
      colorlinks=true,
      urlcolor=blue,
      linkcolor=darkorange,
      citecolor=darkgreen,
      }
    % Slightly bigger margins than the latex defaults
    
    \geometry{verbose,tmargin=1in,bmargin=1in,lmargin=1in,rmargin=1in}
    
    

    \begin{document}
    
    
    \maketitle
    \author
    
    

    
    Start slide show with: \$ ipython nbconvert --to slides --post serve
01\_introduction\_i\_slides.ipynb

    \begin{Verbatim}[commandchars=\\\{\}]
{\color{incolor}In [{\color{incolor}1}]:} \PY{k+kn}{from} \PY{n+nn}{\PYZus{}\PYZus{}future\PYZus{}\PYZus{}} \PY{k+kn}{import} \PY{n}{print\PYZus{}function}\PY{p}{,} \PY{n}{division} \PY{c}{\PYZsh{} if using python 2}
\end{Verbatim}

    \section{Introduction to Python}\label{introduction-to-python}


Highly adapted from: \href{http://www.cs.sandia.gov/~rmuller/}{Rick
Muller}, \href{http://nbviewer.jupyter.org/gist/rpmuller/5920182}{A
Crash Course in Python for Scientists}

\noindent
\href{http://creativecommons.org/licenses/by-sa/3.0/deed.en_US}{Creative
Commons Attribution-ShareAlike 3.0 Unported License}.

    \subsection{Why Python?}\label{why-python}

\begin{itemize}
\item
  well documented
\item
  used by a large and growing number scientists
\item
  runs almost everywhere
\end{itemize}

    \begin{itemize}
\item
  is open
\item
  plays well with others
\end{itemize}

    \subsection{About python}\label{about-python}

\begin{itemize}
\item
  mature language, first released in 1991 by Guido van Rossum

  \begin{itemize}
  \item
    python 2 (2000)
  \item
    python 3 (2008)
  \end{itemize}
\item
  emphasizes readability

  \begin{itemize}
  \item
    identation structure, instead of braces
  \item
    simple English keywords, instead of punctuations
  \end{itemize}
\item
  python is fun

  \begin{itemize}
  \itemsep1pt\parskip0pt\parsep0pt
  \item
    \href{https://www.python.org/doc/humor/}{Python Humor}
  \end{itemize}
\end{itemize}

    picture taken from
\href{http://www.pbs.org/montypython/bios.html}{www.pbs.org}

    \subsubsection{python 2 and python 3}\label{python-2-and-python-3}

\begin{itemize}
\item
  \texttt{/} is no longer floor division (\texttt{//})
  \texttt{from \_\_future\_\_ import division}
\item
  print function \texttt{print\_function}
\item
  \texttt{raw\_input()} was renamed as \texttt{input()}

  \begin{itemize}
  \itemsep1pt\parskip0pt\parsep0pt
  \item
    the old \texttt{input()} was removed, instead use:
    \texttt{eval(input())}
  \end{itemize}
\end{itemize}

\href{http://python-future.org/compatible_idioms.html}{Write compatible
code}

    \section{Getting started: variables}\label{getting-started-variables}

    \subsection{Using Python as a
calculator}\label{using-python-as-a-calculator}

    Many of the things I used to use a calculator for, I now use Python for:

    \begin{Verbatim}[commandchars=\\\{\}]
{\color{incolor}In [{\color{incolor}2}]:} \PY{l+m+mi}{2}\PY{o}{+}\PY{l+m+mi}{2}
\end{Verbatim}

            \begin{Verbatim}[commandchars=\\\{\}]
{\color{outcolor}Out[{\color{outcolor}2}]:} 4
\end{Verbatim}
        
    \begin{Verbatim}[commandchars=\\\{\}]
{\color{incolor}In [{\color{incolor}3}]:} \PY{p}{(}\PY{l+m+mi}{50}\PY{o}{\PYZhy{}}\PY{l+m+mi}{5}\PY{o}{*}\PY{l+m+mi}{6}\PY{p}{)}\PY{o}{/}\PY{l+m+mi}{4}
\end{Verbatim}

            \begin{Verbatim}[commandchars=\\\{\}]
{\color{outcolor}Out[{\color{outcolor}3}]:} 5.0
\end{Verbatim}
        
    \subsubsection{Arithmetic operators}\label{arithmetic-operators}

\begin{longtable}[c]{@{}ll@{}}
\toprule\addlinespace
Symbol & Task Performed
\\\addlinespace
\midrule\endhead
+ & Addition
\\\addlinespace
- & Subtraction
\\\addlinespace
/ & division
\\\addlinespace
\% & mod
\\\addlinespace
* & multiplication
\\\addlinespace
// & floor division
\\\addlinespace
** & to the power of
\\\addlinespace
\bottomrule
\end{longtable}

    import the math library

    \begin{Verbatim}[commandchars=\\\{\}]
{\color{incolor}In [{\color{incolor}1}]:} \PY{k+kn}{import} \PY{n+nn}{math}
        \PY{n}{math}\PY{o}{.}\PY{n}{sqrt}\PY{p}{(}\PY{l+m+mi}{81}\PY{p}{)}
\end{Verbatim}

            \begin{Verbatim}[commandchars=\\\{\}]
{\color{outcolor}Out[{\color{outcolor}1}]:} 9.0
\end{Verbatim}
        
    You can define variables using the equals (=) sign:

    \begin{Verbatim}[commandchars=\\\{\}]
{\color{incolor}In [{\color{incolor}2}]:} \PY{n}{width} \PY{o}{=} \PY{l+m+mi}{20}
        \PY{n}{length} \PY{o}{=} \PY{l+m+mi}{30}
        \PY{n}{area} \PY{o}{=} \PY{n}{length}\PY{o}{*}\PY{n}{width}
        \PY{n}{area}
\end{Verbatim}

            \begin{Verbatim}[commandchars=\\\{\}]
{\color{outcolor}Out[{\color{outcolor}2}]:} 600
\end{Verbatim}
        
    If you try to access a variable that you haven't yet defined, you get an
error:

    \begin{Verbatim}[commandchars=\\\{\}]
{\color{incolor}In [{\color{incolor}3}]:} \PY{n}{volume}
\end{Verbatim}

    \begin{Verbatim}[commandchars=\\\{\}]

        ---------------------------------------------------------------------------

        NameError                                 Traceback (most recent call last)

        <ipython-input-3-0c7fc58f9268> in <module>()
    ----> 1 volume
    

        NameError: name 'volume' is not defined

    \end{Verbatim}

    and you need to define it:

    \begin{Verbatim}[commandchars=\\\{\}]
{\color{incolor}In [{\color{incolor}4}]:} \PY{n}{depth} \PY{o}{=} \PY{l+m+mi}{10}
        \PY{n}{volume} \PY{o}{=} \PY{n}{area}\PY{o}{*}\PY{n}{depth}
        \PY{n}{volume}
\end{Verbatim}

            \begin{Verbatim}[commandchars=\\\{\}]
{\color{outcolor}Out[{\color{outcolor}4}]:} 6000
\end{Verbatim}
        
    You can name a variable \emph{almost} anything you want. It needs to
start with an alphabetical character or ``\_'', can contain alphanumeric
charcters plus underscores (``\_''). Certain words, however, are
reserved for the language:

\begin{verbatim}
and, as, assert, break, class, continue, def, del, elif, else, except, 
exec, finally, for, from, global, if, import, in, is, lambda, not, or,
pass, print, raise, return, try, while, with, yield
\end{verbatim}

Trying to define a variable using one of these will result in a syntax
error:

    \begin{Verbatim}[commandchars=\\\{\}]
{\color{incolor}In [{\color{incolor}5}]:} \PY{k}{return} \PY{o}{=} \PY{l+m+mi}{0}
\end{Verbatim}

    \begin{Verbatim}[commandchars=\\\{\}]

          File "<ipython-input-5-2b99136d4ec6>", line 1
        return = 0
               \^{}
    SyntaxError: invalid syntax


    \end{Verbatim}

    \begin{Verbatim}[commandchars=\\\{\}]
{\color{incolor}In [{\color{incolor}6}]:} \PY{n}{whos}
\end{Verbatim}

    \begin{Verbatim}[commandchars=\\\{\}]
Variable   Type      Data/Info
------------------------------
area       int       600
depth      int       10
length     int       30
math       module    <module 'math' (built-in)>
volume     int       6000
width      int       20
    \end{Verbatim}

    \subsection{Strings}\label{strings}

    \emph{Strings} are lists of printable characters, and can be defined
using either single quotes

    \begin{Verbatim}[commandchars=\\\{\}]
{\color{incolor}In [{\color{incolor}10}]:} \PY{l+s}{\PYZsq{}}\PY{l+s}{Hello, World!}\PY{l+s}{\PYZsq{}}
\end{Verbatim}

            \begin{Verbatim}[commandchars=\\\{\}]
{\color{outcolor}Out[{\color{outcolor}10}]:} 'Hello, World!'
\end{Verbatim}
        
    or double quotes

    \begin{Verbatim}[commandchars=\\\{\}]
{\color{incolor}In [{\color{incolor}11}]:} \PY{l+s}{\PYZdq{}}\PY{l+s}{Hello, World!}\PY{l+s}{\PYZdq{}}
\end{Verbatim}

            \begin{Verbatim}[commandchars=\\\{\}]
{\color{outcolor}Out[{\color{outcolor}11}]:} 'Hello, World!'
\end{Verbatim}
        
    But not both at the same time, unless you want one of the symbols to be
part of the string.

    \begin{Verbatim}[commandchars=\\\{\}]
{\color{incolor}In [{\color{incolor}12}]:} \PY{l+s}{\PYZdq{}}\PY{l+s}{He}\PY{l+s}{\PYZsq{}}\PY{l+s}{s a Rebel}\PY{l+s}{\PYZdq{}}
\end{Verbatim}

            \begin{Verbatim}[commandchars=\\\{\}]
{\color{outcolor}Out[{\color{outcolor}12}]:} "He's a Rebel"
\end{Verbatim}
        
    \begin{Verbatim}[commandchars=\\\{\}]
{\color{incolor}In [{\color{incolor}13}]:} \PY{l+s}{\PYZsq{}}\PY{l+s}{She asked, }\PY{l+s}{\PYZdq{}}\PY{l+s}{How are you today?}\PY{l+s}{\PYZdq{}}\PY{l+s}{\PYZsq{}}
\end{Verbatim}

            \begin{Verbatim}[commandchars=\\\{\}]
{\color{outcolor}Out[{\color{outcolor}13}]:} 'She asked, "How are you today?"'
\end{Verbatim}
        
    Just like the other two data objects we're familiar with (ints and
floats), you can assign a string to a variable

    \begin{Verbatim}[commandchars=\\\{\}]
{\color{incolor}In [{\color{incolor}14}]:} \PY{n}{greeting} \PY{o}{=} \PY{l+s}{\PYZdq{}}\PY{l+s}{Hello, World!}\PY{l+s}{\PYZdq{}}
\end{Verbatim}

    The \textbf{print} function is often used for printing character
strings:

    \begin{Verbatim}[commandchars=\\\{\}]
{\color{incolor}In [{\color{incolor}15}]:} \PY{k}{print}\PY{p}{(}\PY{n}{greeting}\PY{p}{)}
\end{Verbatim}

    \begin{Verbatim}[commandchars=\\\{\}]
Hello, World!
    \end{Verbatim}

    You can use the + operator to concatenate strings together:

    \begin{Verbatim}[commandchars=\\\{\}]
{\color{incolor}In [{\color{incolor}16}]:} \PY{n}{statement} \PY{o}{=} \PY{l+s}{\PYZdq{}}\PY{l+s}{Hello,}\PY{l+s}{\PYZdq{}} \PY{o}{+} \PY{l+s}{\PYZdq{}}\PY{l+s}{World!}\PY{l+s}{\PYZdq{}}
         \PY{k}{print}\PY{p}{(}\PY{n}{statement}\PY{p}{)}
\end{Verbatim}

    \begin{Verbatim}[commandchars=\\\{\}]
Hello,World!
    \end{Verbatim}

    Don't forget the space between the strings, if you want one there.

    \begin{Verbatim}[commandchars=\\\{\}]
{\color{incolor}In [{\color{incolor}17}]:} \PY{n}{statement} \PY{o}{=} \PY{l+s}{\PYZdq{}}\PY{l+s}{Hello, }\PY{l+s}{\PYZdq{}} \PY{o}{+} \PY{l+s}{\PYZdq{}}\PY{l+s}{World!}\PY{l+s}{\PYZdq{}}
         \PY{k}{print}\PY{p}{(}\PY{n}{statement}\PY{p}{)}
\end{Verbatim}

    \begin{Verbatim}[commandchars=\\\{\}]
Hello, World!
    \end{Verbatim}

    You can use + to concatenate multiple strings in a single statement:

    \begin{Verbatim}[commandchars=\\\{\}]
{\color{incolor}In [{\color{incolor}18}]:} \PY{k}{print}\PY{p}{(} \PY{l+s}{\PYZdq{}}\PY{l+s}{This }\PY{l+s}{\PYZdq{}} \PY{o}{+} \PY{l+s}{\PYZdq{}}\PY{l+s}{is }\PY{l+s}{\PYZdq{}} \PY{o}{+} \PY{l+s}{\PYZdq{}}\PY{l+s}{a }\PY{l+s}{\PYZdq{}} \PY{o}{+} \PY{l+s}{\PYZdq{}}\PY{l+s}{longer }\PY{l+s}{\PYZdq{}} \PY{o}{+} \PY{l+s}{\PYZdq{}}\PY{l+s}{statement.}\PY{l+s}{\PYZdq{}}\PY{p}{)}
\end{Verbatim}

    \begin{Verbatim}[commandchars=\\\{\}]
This is a longer statement.
    \end{Verbatim}

    If you have a lot of words to concatenate together, there are other,
more efficient ways to do this. But this is fine for linking a few
strings together.

    It can also print data types other than strings:

    \begin{Verbatim}[commandchars=\\\{\}]
{\color{incolor}In [{\color{incolor}19}]:} \PY{k}{print}\PY{p}{(} \PY{l+s}{\PYZdq{}}\PY{l+s}{The area is \PYZob{}\PYZcb{}}\PY{l+s}{\PYZdq{}}\PY{o}{.}\PY{n}{format}\PY{p}{(}\PY{n}{area}\PY{p}{)}\PY{p}{)}
         \PY{k}{print}\PY{p}{(} \PY{l+s}{\PYZdq{}}\PY{l+s}{The area is \PYZob{}0:.1f\PYZcb{}}\PY{l+s}{\PYZdq{}}\PY{o}{.}\PY{n}{format}\PY{p}{(}\PY{n}{area}\PY{p}{)}\PY{p}{)}
\end{Verbatim}

    \begin{Verbatim}[commandchars=\\\{\}]
The area is 600
The area is 600.0
    \end{Verbatim}

    In the above snipped, the number 600 (stored in the variable ``area'')
is converted into a string before being printed out.

    \subsection{Lists}\label{lists}

Very often in a programming language, one wants to keep a group of
similar items together. Python does this using a data type called
\textbf{lists}.

    \begin{Verbatim}[commandchars=\\\{\}]
{\color{incolor}In [{\color{incolor}20}]:} \PY{n}{days\PYZus{}of\PYZus{}the\PYZus{}week} \PY{o}{=} \PY{p}{[}\PY{l+s}{\PYZdq{}}\PY{l+s}{Monday}\PY{l+s}{\PYZdq{}}\PY{p}{,}\PY{l+s}{\PYZdq{}}\PY{l+s}{Tuesday}\PY{l+s}{\PYZdq{}}\PY{p}{,}\PY{l+s}{\PYZdq{}}\PY{l+s}{Wednesday}\PY{l+s}{\PYZdq{}}\PY{p}{,}\PY{l+s}{\PYZdq{}}\PY{l+s}{Thursday}\PY{l+s}{\PYZdq{}}\PY{p}{,}\PY{l+s}{\PYZdq{}}\PY{l+s}{Friday}\PY{l+s}{\PYZdq{}}\PY{p}{,}\PY{l+s}{\PYZdq{}}\PY{l+s}{Saturday}\PY{l+s}{\PYZdq{}}\PY{p}{]}
\end{Verbatim}

    You can access members of the list using the \textbf{index} of that
item:

    \begin{Verbatim}[commandchars=\\\{\}]
{\color{incolor}In [{\color{incolor}21}]:} \PY{n}{days\PYZus{}of\PYZus{}the\PYZus{}week}\PY{p}{[}\PY{l+m+mi}{2}\PY{p}{]}
\end{Verbatim}

            \begin{Verbatim}[commandchars=\\\{\}]
{\color{outcolor}Out[{\color{outcolor}21}]:} 'Wednesday'
\end{Verbatim}
        
    Python lists, like C, but unlike Fortran, use 0 as the index of the
first element of a list. Thus, in this example, the 0 element is
``Sunday'', 1 is ``Monday'', and so on. If you need to access the
\emph{n}th element from the end of the list, you can use a negative
index. For example, the -1 element of a list is the last element:

    \begin{Verbatim}[commandchars=\\\{\}]
{\color{incolor}In [{\color{incolor}22}]:} \PY{n}{days\PYZus{}of\PYZus{}the\PYZus{}week}\PY{p}{[}\PY{o}{\PYZhy{}}\PY{l+m+mi}{1}\PY{p}{]}
\end{Verbatim}

            \begin{Verbatim}[commandchars=\\\{\}]
{\color{outcolor}Out[{\color{outcolor}22}]:} 'Saturday'
\end{Verbatim}
        
    You can add additional items to the list using the .append() command:

    \begin{Verbatim}[commandchars=\\\{\}]
{\color{incolor}In [{\color{incolor}23}]:} \PY{n}{days\PYZus{}of\PYZus{}the\PYZus{}week}\PY{o}{.}\PY{n}{append}\PY{p}{(}\PY{l+s}{\PYZdq{}}\PY{l+s}{Sunday}\PY{l+s}{\PYZdq{}}\PY{p}{)}
         \PY{k}{print}\PY{p}{(}\PY{n}{days\PYZus{}of\PYZus{}the\PYZus{}week}\PY{p}{)}
\end{Verbatim}

    \begin{Verbatim}[commandchars=\\\{\}]
['Monday', 'Tuesday', 'Wednesday', 'Thursday', 'Friday', 'Saturday', 'Sunday']
    \end{Verbatim}

    The \textbf{range()} command is a convenient way to make sequential
lists of numbers:

    \begin{Verbatim}[commandchars=\\\{\}]
{\color{incolor}In [{\color{incolor}24}]:} \PY{n+nb}{range}\PY{p}{(}\PY{l+m+mi}{10}\PY{p}{)}
\end{Verbatim}

            \begin{Verbatim}[commandchars=\\\{\}]
{\color{outcolor}Out[{\color{outcolor}24}]:} [0, 1, 2, 3, 4, 5, 6, 7, 8, 9]
\end{Verbatim}
        
    Note that range(n) starts at 0 and gives the sequential list of integers
less than n. If you want to start at a different number, use
range(start,stop)

    \begin{Verbatim}[commandchars=\\\{\}]
{\color{incolor}In [{\color{incolor}25}]:} \PY{n+nb}{range}\PY{p}{(}\PY{l+m+mi}{2}\PY{p}{,}\PY{l+m+mi}{8}\PY{p}{)}
\end{Verbatim}

            \begin{Verbatim}[commandchars=\\\{\}]
{\color{outcolor}Out[{\color{outcolor}25}]:} [2, 3, 4, 5, 6, 7]
\end{Verbatim}
        
    The lists created above with range have a \emph{step} of 1 between
elements. You can also give a fixed step size via a third command:

    \begin{Verbatim}[commandchars=\\\{\}]
{\color{incolor}In [{\color{incolor}26}]:} \PY{n}{evens} \PY{o}{=} \PY{n+nb}{range}\PY{p}{(}\PY{l+m+mi}{0}\PY{p}{,}\PY{l+m+mi}{20}\PY{p}{,}\PY{l+m+mi}{2}\PY{p}{)}
         \PY{n}{evens}
\end{Verbatim}

            \begin{Verbatim}[commandchars=\\\{\}]
{\color{outcolor}Out[{\color{outcolor}26}]:} [0, 2, 4, 6, 8, 10, 12, 14, 16, 18]
\end{Verbatim}
        
    \begin{Verbatim}[commandchars=\\\{\}]
{\color{incolor}In [{\color{incolor}27}]:} \PY{n}{evens}\PY{p}{[}\PY{l+m+mi}{3}\PY{p}{]}
\end{Verbatim}

            \begin{Verbatim}[commandchars=\\\{\}]
{\color{outcolor}Out[{\color{outcolor}27}]:} 6
\end{Verbatim}
        
    Lists do not have to hold the same data type. For example,

    \begin{Verbatim}[commandchars=\\\{\}]
{\color{incolor}In [{\color{incolor}28}]:} \PY{p}{[}\PY{l+s}{\PYZdq{}}\PY{l+s}{Today}\PY{l+s}{\PYZdq{}}\PY{p}{,}\PY{l+m+mi}{7}\PY{p}{,}\PY{l+m+mf}{99.3}\PY{p}{,}\PY{l+s}{\PYZdq{}}\PY{l+s}{\PYZdq{}}\PY{p}{]}
\end{Verbatim}

            \begin{Verbatim}[commandchars=\\\{\}]
{\color{outcolor}Out[{\color{outcolor}28}]:} ['Today', 7, 99.3, '']
\end{Verbatim}
        
    However, it's good (but not essential) to use lists for similar objects
that are somehow logically connected. If you want to group different
data types together into a composite data object, it's best to use
\textbf{tuples}, which we will learn about below.

    You can find out how long a list is using the \textbf{len()} command:

    \begin{Verbatim}[commandchars=\\\{\}]
{\color{incolor}In [{\color{incolor}29}]:} \PY{n}{help}\PY{p}{(}\PY{n+nb}{len}\PY{p}{)}
\end{Verbatim}

    \begin{Verbatim}[commandchars=\\\{\}]
Help on built-in function len in module \_\_builtin\_\_:

len({\ldots})
    len(object) -> integer
    
    Return the number of items of a sequence or mapping.
    \end{Verbatim}

    \begin{Verbatim}[commandchars=\\\{\}]
{\color{incolor}In [{\color{incolor}30}]:} \PY{n+nb}{len}\PY{p}{(}\PY{n}{evens}\PY{p}{)}
\end{Verbatim}

            \begin{Verbatim}[commandchars=\\\{\}]
{\color{outcolor}Out[{\color{outcolor}30}]:} 10
\end{Verbatim}
        
    \section{Repeating Actions with Loops: Iteration, Indentation, and
blocks}\label{repeating-actions-with-loops-iteration-indentation-and-blocks}

    One of the most useful things you can do with lists is to \emph{iterate}
through them, i.e.~to go through each element one at a time. To do this
in Python, we use the \textbf{for} statement:

    \begin{Verbatim}[commandchars=\\\{\}]
{\color{incolor}In [{\color{incolor}31}]:} \PY{k}{for} \PY{n}{day} \PY{o+ow}{in} \PY{n}{days\PYZus{}of\PYZus{}the\PYZus{}week}\PY{p}{:}
             \PY{k}{print}\PY{p}{(}\PY{n}{day}\PY{p}{)}
\end{Verbatim}

    \begin{Verbatim}[commandchars=\\\{\}]
Monday
Tuesday
Wednesday
Thursday
Friday
Saturday
Sunday
    \end{Verbatim}

    This code snippet goes through each element of the list called
\textbf{days\_of\_the\_week} and assigns it to the variable
\textbf{day}. It then executes everything in the indented block (in this
case only one line of code, the print statement) using those variable
assignments. When the program has gone through every element of the
list, it exists the block.

(Almost) every programming language defines blocks of code in some way.
In Fortran, one uses END statements (ENDDO, ENDIF, etc.) to define code
blocks. In C, C++, and Perl, one uses curly braces \{\} to define these
blocks.

Python uses a colon (``:''), followed by indentation level to define
code blocks. Everything at a higher level of indentation is taken to be
in the same block. In the above example the block was only a single
line, but we could have had longer blocks as well:

    \begin{Verbatim}[commandchars=\\\{\}]
{\color{incolor}In [{\color{incolor}32}]:} \PY{k}{for} \PY{n}{day} \PY{o+ow}{in} \PY{n}{days\PYZus{}of\PYZus{}the\PYZus{}week}\PY{p}{:}
             \PY{n}{statement} \PY{o}{=} \PY{l+s}{\PYZdq{}}\PY{l+s}{Today is }\PY{l+s}{\PYZdq{}} \PY{o}{+} \PY{n}{day}
             \PY{k}{print}\PY{p}{(} \PY{n}{statement} \PY{p}{)}
\end{Verbatim}

    \begin{Verbatim}[commandchars=\\\{\}]
Today is Monday
Today is Tuesday
Today is Wednesday
Today is Thursday
Today is Friday
Today is Saturday
Today is Sunday
    \end{Verbatim}

    The \textbf{range()} command is particularly useful with the
\textbf{for} statement to execute loops of a specified length:

    \begin{Verbatim}[commandchars=\\\{\}]
{\color{incolor}In [{\color{incolor}33}]:} \PY{k}{for} \PY{n}{i} \PY{o+ow}{in} \PY{n+nb}{range}\PY{p}{(}\PY{l+m+mi}{20}\PY{p}{)}\PY{p}{:}
             \PY{k}{print}\PY{p}{(}\PY{l+s}{\PYZdq{}}\PY{l+s}{The square of }\PY{l+s}{\PYZdq{}}\PY{p}{,}\PY{n}{i}\PY{p}{,}\PY{l+s}{\PYZdq{}}\PY{l+s}{ is }\PY{l+s}{\PYZdq{}}\PY{p}{,}\PY{n}{i}\PY{o}{*}\PY{n}{i}\PY{p}{)}
\end{Verbatim}

    \begin{Verbatim}[commandchars=\\\{\}]
The square of  0  is  0
The square of  1  is  1
The square of  2  is  4
The square of  3  is  9
The square of  4  is  16
The square of  5  is  25
The square of  6  is  36
The square of  7  is  49
The square of  8  is  64
The square of  9  is  81
The square of  10  is  100
The square of  11  is  121
The square of  12  is  144
The square of  13  is  169
The square of  14  is  196
The square of  15  is  225
The square of  16  is  256
The square of  17  is  289
The square of  18  is  324
The square of  19  is  361
    \end{Verbatim}

    Lists and strings have something in common that you might not suspect:
they can both be treated as sequences, i.e. you can iterate through its
elements.

    \begin{Verbatim}[commandchars=\\\{\}]
{\color{incolor}In [{\color{incolor}34}]:} \PY{k}{for} \PY{n}{letter} \PY{o+ow}{in} \PY{l+s}{\PYZdq{}}\PY{l+s}{Sunday}\PY{l+s}{\PYZdq{}}\PY{p}{:}
             \PY{k}{print}\PY{p}{(}\PY{n}{letter}\PY{p}{)}
\end{Verbatim}

    \begin{Verbatim}[commandchars=\\\{\}]
S
u
n
d
a
y
    \end{Verbatim}

    \subsection{Slicing \texttt{{[}:{]}}}\label{slicing}

    A useful way to iterate over the elements of a sequence is by using
\textbf{\texttt{{[}:{]}}}. We already know that we can use
\emph{indexing} to get the first element of a list:

    \begin{Verbatim}[commandchars=\\\{\}]
{\color{incolor}In [{\color{incolor}35}]:} \PY{n}{days\PYZus{}of\PYZus{}the\PYZus{}week}\PY{p}{[}\PY{l+m+mi}{0}\PY{p}{]}
\end{Verbatim}

            \begin{Verbatim}[commandchars=\\\{\}]
{\color{outcolor}Out[{\color{outcolor}35}]:} 'Monday'
\end{Verbatim}
        
    If we want the list containing the first two elements of a list, we can
do this via

    \begin{Verbatim}[commandchars=\\\{\}]
{\color{incolor}In [{\color{incolor}36}]:} \PY{n}{days\PYZus{}of\PYZus{}the\PYZus{}week}\PY{p}{[}\PY{l+m+mi}{0}\PY{p}{:}\PY{l+m+mi}{2}\PY{p}{]}
\end{Verbatim}

            \begin{Verbatim}[commandchars=\\\{\}]
{\color{outcolor}Out[{\color{outcolor}36}]:} ['Monday', 'Tuesday']
\end{Verbatim}
        
    or simply

    \begin{Verbatim}[commandchars=\\\{\}]
{\color{incolor}In [{\color{incolor}37}]:} \PY{n}{days\PYZus{}of\PYZus{}the\PYZus{}week}\PY{p}{[}\PY{p}{:}\PY{l+m+mi}{2}\PY{p}{]}
\end{Verbatim}

            \begin{Verbatim}[commandchars=\\\{\}]
{\color{outcolor}Out[{\color{outcolor}37}]:} ['Monday', 'Tuesday']
\end{Verbatim}
        
    If we want the last items of the list, we can do this with negative
slicing:

    \begin{Verbatim}[commandchars=\\\{\}]
{\color{incolor}In [{\color{incolor}38}]:} \PY{n}{days\PYZus{}of\PYZus{}the\PYZus{}week}\PY{p}{[}\PY{o}{\PYZhy{}}\PY{l+m+mi}{2}\PY{p}{:}\PY{p}{]}
\end{Verbatim}

            \begin{Verbatim}[commandchars=\\\{\}]
{\color{outcolor}Out[{\color{outcolor}38}]:} ['Saturday', 'Sunday']
\end{Verbatim}
        
    which is somewhat logically consistent with negative indices accessing
the last elements of the list.

You can do:

    \begin{Verbatim}[commandchars=\\\{\}]
{\color{incolor}In [{\color{incolor}39}]:} \PY{n}{workdays} \PY{o}{=} \PY{n}{days\PYZus{}of\PYZus{}the\PYZus{}week}\PY{p}{[}\PY{l+m+mi}{1}\PY{p}{:}\PY{l+m+mi}{6}\PY{p}{]}
         \PY{k}{print}\PY{p}{(}\PY{n}{workdays}\PY{p}{)}
\end{Verbatim}

    \begin{Verbatim}[commandchars=\\\{\}]
['Tuesday', 'Wednesday', 'Thursday', 'Friday', 'Saturday']
    \end{Verbatim}

    Since strings are sequences, you can also do this to them:

\emph{exercise!}

    \begin{Verbatim}[commandchars=\\\{\}]
{\color{incolor}In [{\color{incolor}40}]:} \PY{n}{day} \PY{o}{=} \PY{l+s}{\PYZdq{}}\PY{l+s}{Sunday}\PY{l+s}{\PYZdq{}}
         \PY{n}{abbreviation} \PY{o}{=} \PY{n}{day}\PY{p}{[}\PY{p}{:}\PY{l+m+mi}{3}\PY{p}{]}
         \PY{k}{print}\PY{p}{(}\PY{n}{abbreviation}\PY{p}{)}
\end{Verbatim}

    \begin{Verbatim}[commandchars=\\\{\}]
Sun
    \end{Verbatim}

    If we really want to get fancy, we can pass a third element into the
slice, which specifies a step length (just like a third argument to the
\textbf{range()} function specifies the step):

    \begin{Verbatim}[commandchars=\\\{\}]
{\color{incolor}In [{\color{incolor}41}]:} \PY{n}{numbers} \PY{o}{=} \PY{n+nb}{range}\PY{p}{(}\PY{l+m+mi}{0}\PY{p}{,}\PY{l+m+mi}{40}\PY{p}{)}
         \PY{n}{evens} \PY{o}{=} \PY{n}{numbers}\PY{p}{[}\PY{l+m+mi}{2}\PY{p}{:}\PY{p}{:}\PY{l+m+mi}{2}\PY{p}{]}
         \PY{n}{evens}
\end{Verbatim}

            \begin{Verbatim}[commandchars=\\\{\}]
{\color{outcolor}Out[{\color{outcolor}41}]:} [2, 4, 6, 8, 10, 12, 14, 16, 18, 20, 22, 24, 26, 28, 30, 32, 34, 36, 38]
\end{Verbatim}
        
    Note that in this example I was even able to omit the second argument,
so that the slice started at 2, went to the end of the list, and took
every second element, to generate the list of even numbers less that 40.

    \section{Making choices}\label{making-choices}

True, False, conditions and control statements (\emph{if}, \emph{while})

    \subsection{\texttt{True}, \texttt{False}}\label{true-false}

\texttt{False}: zero, None, empty container or object

\texttt{True}: non-zero numbers, non-empty objects

\subsection{Boolean logic}\label{boolean-logic}

\begin{longtable}[c]{@{}lc@{}}
\toprule\addlinespace
Operator & True, if
\\\addlinespace
\midrule\endhead
a and b & a and b are True
\\\addlinespace
a or b & a or b (or both) are True
\\\addlinespace
not a & a is False
\\\addlinespace
\bottomrule
\end{longtable}

    \begin{Verbatim}[commandchars=\\\{\}]
{\color{incolor}In [{\color{incolor}42}]:} \PY{n}{name} \PY{o}{=} \PY{l+s}{\PYZsq{}}\PY{l+s}{Bob}\PY{l+s}{\PYZsq{}}
         \PY{n}{age} \PY{o}{=} \PY{l+m+mi}{98}
         \PY{k}{print}\PY{p}{(} \PY{n}{name}\PY{o}{==}\PY{l+s}{\PYZsq{}}\PY{l+s}{bob}\PY{l+s}{\PYZsq{}} \PY{o+ow}{and} \PY{n}{age}\PY{o}{\PYZlt{}}\PY{l+m+mi}{99}\PY{p}{)}
         \PY{k}{print}\PY{p}{(} \PY{n}{age} \PY{o}{\PYZlt{}} \PY{l+m+mi}{99} \PY{o+ow}{and} \PY{p}{(}\PY{n}{age} \PY{o}{\PYZgt{}} \PY{l+m+mi}{1} \PY{o+ow}{or} \PY{n}{name}\PY{o}{==}\PY{l+s}{\PYZsq{}}\PY{l+s}{bob}\PY{l+s}{\PYZsq{}}\PY{p}{)} \PY{p}{)}
\end{Verbatim}

    \begin{Verbatim}[commandchars=\\\{\}]
False
True
    \end{Verbatim}

    \begin{Verbatim}[commandchars=\\\{\}]
{\color{incolor}In [{\color{incolor}43}]:} \PY{l+m+mi}{10} \PY{o+ow}{and} \PY{l+m+mf}{1.2} \PY{o+ow}{and} \PY{l+m+mi}{3} 
\end{Verbatim}

            \begin{Verbatim}[commandchars=\\\{\}]
{\color{outcolor}Out[{\color{outcolor}43}]:} 3
\end{Verbatim}
        
    \subsubsection{Comparison operators}\label{comparison-operators}

\begin{longtable}[c]{@{}lc@{}}
\toprule\addlinespace
Operator & True, if
\\\addlinespace
\midrule\endhead
a == b & a equals b
\\\addlinespace
a \textgreater{} b & a is larger than b
\\\addlinespace
a \textless{} b & a is smaller than b
\\\addlinespace
a \textgreater{}= b & a is larger than b or equals b
\\\addlinespace
a \textless{}= b & a is smaller than b or equals b
\\\addlinespace
a != b & a and b are unequal
\\\addlinespace
a is b & a is the same object as b
\\\addlinespace
a is not b & a is not the same object as b
\\\addlinespace
\bottomrule
\end{longtable}

    We see a few other boolean operators here, all of which should be
self-explanatory. Less than, equality, non-equality, and so on.

Particularly interesting is the 1 == 1.0 test, which is true, since even
though the two objects are different data types (integer and floating
point number), they have the same \emph{value}.

    \begin{Verbatim}[commandchars=\\\{\}]
{\color{incolor}In [{\color{incolor}44}]:} \PY{l+m+mi}{1} \PY{o}{==} \PY{l+m+mf}{1.0}
\end{Verbatim}

            \begin{Verbatim}[commandchars=\\\{\}]
{\color{outcolor}Out[{\color{outcolor}44}]:} True
\end{Verbatim}
        
    \texttt{is}, tests whether two objects are the same object:

    \begin{Verbatim}[commandchars=\\\{\}]
{\color{incolor}In [{\color{incolor}45}]:} \PY{l+m+mi}{1} \PY{o+ow}{is} \PY{l+m+mf}{1.0}
\end{Verbatim}

            \begin{Verbatim}[commandchars=\\\{\}]
{\color{outcolor}Out[{\color{outcolor}45}]:} False
\end{Verbatim}
        
    \begin{Verbatim}[commandchars=\\\{\}]
{\color{incolor}In [{\color{incolor}46}]:} \PY{l+m+mi}{1} \PY{o}{!=} \PY{l+m+mi}{0}
\end{Verbatim}

            \begin{Verbatim}[commandchars=\\\{\}]
{\color{outcolor}Out[{\color{outcolor}46}]:} True
\end{Verbatim}
        
    In addition to int and float, many other data types can be compared as
well:

    \begin{Verbatim}[commandchars=\\\{\}]
{\color{incolor}In [{\color{incolor}47}]:} \PY{k}{print}\PY{p}{(}\PY{l+s}{\PYZsq{}}\PY{l+s}{color}\PY{l+s}{\PYZsq{}} \PY{o}{==} \PY{l+s}{\PYZsq{}}\PY{l+s}{color}\PY{l+s}{\PYZsq{}}\PY{p}{)}
         \PY{k}{print}\PY{p}{(}\PY{l+s}{\PYZsq{}}\PY{l+s}{color}\PY{l+s}{\PYZsq{}} \PY{o}{!=} \PY{l+s}{\PYZsq{}}\PY{l+s}{colour}\PY{l+s}{\PYZsq{}}\PY{p}{)}
\end{Verbatim}

    \begin{Verbatim}[commandchars=\\\{\}]
True
True
    \end{Verbatim}

    \subsubsection{If \ldots{} else \ldots{}}\label{if-else}

\begin{verbatim}
number_of_people = 3

if number_of_people < 5:
    print('Sorry, you need five or more people to play this game.')
\end{verbatim}

    \begin{Verbatim}[commandchars=\\\{\}]
{\color{incolor}In [{\color{incolor}48}]:} \PY{n}{number\PYZus{}of\PYZus{}people} \PY{o}{=} \PY{l+m+mi}{3}
         
         \PY{k}{if} \PY{n}{number\PYZus{}of\PYZus{}people} \PY{o}{\PYZlt{}} \PY{l+m+mi}{5}\PY{p}{:}
             \PY{k}{print}\PY{p}{(}\PY{l+s}{\PYZsq{}}\PY{l+s}{Sorry, you need five or more people to play this game.}\PY{l+s}{\PYZsq{}}\PY{p}{)}
\end{Verbatim}

    \begin{Verbatim}[commandchars=\\\{\}]
Sorry, you need five or more people to play this game.
    \end{Verbatim}

    \begin{Verbatim}[commandchars=\\\{\}]
{\color{incolor}In [{\color{incolor}49}]:} \PY{n}{number\PYZus{}of\PYZus{}people} \PY{o}{=} \PY{l+m+mi}{6}
         
         \PY{k}{if} \PY{n}{number\PYZus{}of\PYZus{}people} \PY{o}{\PYZlt{}} \PY{l+m+mi}{5}\PY{p}{:}
             \PY{k}{print}\PY{p}{(}\PY{l+s}{\PYZsq{}}\PY{l+s}{Not enough people to play this game.}\PY{l+s}{\PYZsq{}}\PY{p}{)}
         \PY{k}{else}\PY{p}{:}
             \PY{k}{print}\PY{p}{(}\PY{l+s}{\PYZsq{}}\PY{l+s}{Thats enough. Enjoy!}\PY{l+s}{\PYZsq{}}\PY{p}{)}
\end{Verbatim}

    \begin{Verbatim}[commandchars=\\\{\}]
Thats enough. Enjoy!
    \end{Verbatim}

    \begin{Verbatim}[commandchars=\\\{\}]
{\color{incolor}In [{\color{incolor}50}]:} \PY{n}{number\PYZus{}of\PYZus{}people} \PY{o}{=} \PY{l+m+mi}{6}
         
         \PY{k}{if} \PY{n}{number\PYZus{}of\PYZus{}people} \PY{o}{\PYZlt{}} \PY{l+m+mi}{5}\PY{p}{:}
             \PY{k}{print}\PY{p}{(}\PY{l+s}{\PYZsq{}}\PY{l+s}{Not enough people to play this game.}\PY{l+s}{\PYZsq{}}\PY{p}{)}
         \PY{k}{elif} \PY{n}{number\PYZus{}of\PYZus{}people} \PY{o}{\PYZlt{}} \PY{l+m+mi}{10}\PY{p}{:}
             \PY{k}{print}\PY{p}{(}\PY{l+s}{\PYZdq{}}\PY{l+s}{More would be better, but it}\PY{l+s}{\PYZsq{}}\PY{l+s}{s sufficient.}\PY{l+s}{\PYZdq{}}\PY{p}{)}
         \PY{k}{elif} \PY{n}{number\PYZus{}of\PYZus{}people} \PY{o}{\PYZlt{}} \PY{l+m+mi}{20}\PY{p}{:}
             \PY{k}{print}\PY{p}{(}\PY{l+s}{\PYZsq{}}\PY{l+s}{Perfect! Enjoy!}\PY{l+s}{\PYZsq{}}\PY{p}{)}
         \PY{k}{elif} \PY{n}{number\PYZus{}of\PYZus{}people} \PY{o}{\PYZlt{}} \PY{l+m+mi}{30}\PY{p}{:}
             \PY{k}{print}\PY{p}{(}\PY{l+s}{\PYZsq{}}\PY{l+s}{Less would be better, but it will work somehow.}\PY{l+s}{\PYZsq{}}\PY{p}{)}
         \PY{k}{else}\PY{p}{:}
             \PY{k}{print}\PY{p}{(}\PY{l+s}{\PYZsq{}}\PY{l+s}{Sorry, but more than 30 is too much.}\PY{l+s}{\PYZsq{}}\PY{p}{)}
\end{Verbatim}

    \begin{Verbatim}[commandchars=\\\{\}]
More would be better, but it's sufficient.
    \end{Verbatim}

    \begin{Verbatim}[commandchars=\\\{\}]
{\color{incolor}In [{\color{incolor}51}]:} \PY{n}{x} \PY{o}{=} \PY{l+m+mi}{12}
         
         \PY{c}{\PYZsh{}the long version:}
         \PY{k}{if} \PY{n}{x}\PY{o}{\PYZpc{}}\PY{k}{2}==0:
             \PY{n}{message} \PY{o}{=} \PY{l+s}{\PYZdq{}}\PY{l+s}{Even.}\PY{l+s}{\PYZdq{}}
         \PY{k}{else}\PY{p}{:}
             \PY{n}{message} \PY{o}{=} \PY{l+s}{\PYZdq{}}\PY{l+s}{Odd.}\PY{l+s}{\PYZdq{}}
         \PY{k}{print}\PY{p}{(}\PY{n}{message}\PY{p}{)}
         
         \PY{c}{\PYZsh{}the short version:}
         \PY{k}{print}\PY{p}{(} \PY{l+s}{\PYZdq{}}\PY{l+s}{Even.}\PY{l+s}{\PYZdq{}} \PY{k}{if} \PY{n}{x}\PY{o}{\PYZpc{}}\PY{k}{2}==0 else \PYZdq{}Odd.\PYZdq{} ) 
\end{Verbatim}

    \begin{Verbatim}[commandchars=\\\{\}]
Even.
Even.
    \end{Verbatim}

    \subsubsection{While}\label{while}

loop while the condition is True:

    \begin{Verbatim}[commandchars=\\\{\}]
{\color{incolor}In [{\color{incolor}52}]:} \PY{n}{value} \PY{o}{=} \PY{l+m+mi}{17}
         
         \PY{k}{while} \PY{n}{value} \PY{o}{\PYZlt{}} \PY{l+m+mi}{21}\PY{p}{:}
             \PY{k}{print}\PY{p}{(}\PY{n}{value}\PY{p}{)}
             \PY{n}{value} \PY{o}{=} \PY{n}{value} \PY{o}{+} \PY{l+m+mi}{1}
\end{Verbatim}

    \begin{Verbatim}[commandchars=\\\{\}]
17
18
19
20
    \end{Verbatim}

    We can control the information flow with the statements: *
\textbf{\texttt{break}}, breaks the loop * \textbf{\texttt{continue}},
continues with the next iteration

    \begin{Verbatim}[commandchars=\\\{\}]
{\color{incolor}In [{\color{incolor}53}]:} \PY{n}{value} \PY{o}{=} \PY{l+m+mi}{17}
         \PY{n}{max\PYZus{}value} \PY{o}{=} \PY{l+m+mi}{30}
         
         \PY{k}{while} \PY{n+nb+bp}{True}\PY{p}{:}
             \PY{n}{value} \PY{o}{=} \PY{n}{value} \PY{o}{+} \PY{l+m+mi}{1}
             \PY{k}{if} \PY{n}{value} \PY{o}{\PYZgt{}} \PY{n}{max\PYZus{}value}\PY{p}{:}
                 \PY{k}{break}              \PY{c}{\PYZsh{}stop here and escape the while loop}
             \PY{k}{elif} \PY{n}{value}\PY{o}{\PYZpc{}}\PY{k}{2}==0:
                 \PY{k}{continue}           \PY{c}{\PYZsh{}stop here and continue the while loop}
             \PY{k}{print}\PY{p}{(}\PY{n}{value}\PY{p}{)}
\end{Verbatim}

    \begin{Verbatim}[commandchars=\\\{\}]
19
21
23
25
27
29
    \end{Verbatim}

    \textbf{Example of really bad code: }

\begin{verbatim}
value = 0
increment = 3 

while not value == 100:
    value = value + increment
\end{verbatim}

Finishes if \texttt{increment = 1 or 2}\\but if
i\texttt{ncrement = 0 or 3}, the program is trapped in an infinite loop.
To stop it click on \emph{interrupt kernel}

    ** Exercise:** Find the smallest Fibonacci number which is bigger than
1000000.

Definition of Fibonacci numbers: $F_0=0$, $F_1=1$ and
$F_n=F_{n-1}+F_{n-2}$

The first few numbers are: \{0 , 1, 1, 2, 3, 5, 8, 13, 21, 34, 55, 89,
144, \ldots{}\}

    \section{Two more data structures: Tuples and
Dictionaries}\label{two-more-data-structures-tuples-and-dictionaries}

    \subsection{Tuple}\label{tuple}

\begin{itemize}
\item
  sequence object like a list or a string
\item
  constructed by grouping a sequence of objects together with commas,
  either without brackets, or with parentheses:
\end{itemize}

    \begin{Verbatim}[commandchars=\\\{\}]
{\color{incolor}In [{\color{incolor}54}]:} \PY{n}{t} \PY{o}{=} \PY{p}{(}\PY{l+m+mi}{1}\PY{p}{,}\PY{l+m+mi}{2}\PY{p}{,}\PY{l+s}{\PYZsq{}}\PY{l+s}{hi}\PY{l+s}{\PYZsq{}}\PY{p}{,}\PY{l+m+mf}{9.0}\PY{p}{)}
         \PY{n}{t}
\end{Verbatim}

            \begin{Verbatim}[commandchars=\\\{\}]
{\color{outcolor}Out[{\color{outcolor}54}]:} (1, 2, 'hi', 9.0)
\end{Verbatim}
        
    Tuples are like lists, in that you can access the elements using
indices:

    \begin{Verbatim}[commandchars=\\\{\}]
{\color{incolor}In [{\color{incolor}55}]:} \PY{n}{t}\PY{p}{[}\PY{l+m+mi}{1}\PY{p}{]}
\end{Verbatim}

            \begin{Verbatim}[commandchars=\\\{\}]
{\color{outcolor}Out[{\color{outcolor}55}]:} 2
\end{Verbatim}
        
    However, tuples are \emph{immutable}, once created you can't modify
them:

    \begin{Verbatim}[commandchars=\\\{\}]
{\color{incolor}In [{\color{incolor}56}]:} \PY{n}{t}\PY{p}{[}\PY{l+m+mi}{1}\PY{p}{]}\PY{o}{=}\PY{l+m+mi}{77}
\end{Verbatim}

    \begin{Verbatim}[commandchars=\\\{\}]

        ---------------------------------------------------------------------------

        TypeError                                 Traceback (most recent call last)

        <ipython-input-56-03cc8ba9c07d> in <module>()
    ----> 1 t[1]=77
    

        TypeError: 'tuple' object does not support item assignment

    \end{Verbatim}

    Tuples are useful anytime you want to group different pieces of data
together in an object, but don't want to create a full-fledged class
(see below) for them. For example, let's say you want the Cartesian
coordinates of some objects in your program. Tuples are a good way to do
this:

    \begin{Verbatim}[commandchars=\\\{\}]
{\color{incolor}In [{\color{incolor}57}]:} \PY{p}{(}\PY{l+s}{\PYZsq{}}\PY{l+s}{Bob}\PY{l+s}{\PYZsq{}}\PY{p}{,}\PY{l+m+mf}{0.0}\PY{p}{,}\PY{l+m+mf}{21.0}\PY{p}{)}
\end{Verbatim}

            \begin{Verbatim}[commandchars=\\\{\}]
{\color{outcolor}Out[{\color{outcolor}57}]:} ('Bob', 0.0, 21.0)
\end{Verbatim}
        
    The \textbf{\texttt{in}} checks whether an element is contained in a
sequence:

    \begin{Verbatim}[commandchars=\\\{\}]
{\color{incolor}In [{\color{incolor}58}]:} \PY{l+s}{\PYZsq{}}\PY{l+s}{Bob}\PY{l+s}{\PYZsq{}} \PY{o+ow}{in} \PY{p}{(}\PY{l+s}{\PYZsq{}}\PY{l+s}{Bob}\PY{l+s}{\PYZsq{}}\PY{p}{,}\PY{l+m+mf}{0.0}\PY{p}{,}\PY{l+m+mf}{21.0}\PY{p}{)}
\end{Verbatim}

            \begin{Verbatim}[commandchars=\\\{\}]
{\color{outcolor}Out[{\color{outcolor}58}]:} True
\end{Verbatim}
        
    \begin{Verbatim}[commandchars=\\\{\}]
{\color{incolor}In [{\color{incolor}59}]:} \PY{l+m+mi}{2} \PY{o+ow}{in} \PY{p}{[}\PY{l+m+mi}{1}\PY{p}{,}\PY{l+m+mi}{2}\PY{p}{,}\PY{l+m+mi}{3}\PY{p}{,}\PY{l+m+mi}{4}\PY{p}{,}\PY{l+m+mi}{5}\PY{p}{]}
\end{Verbatim}

            \begin{Verbatim}[commandchars=\\\{\}]
{\color{outcolor}Out[{\color{outcolor}59}]:} True
\end{Verbatim}
        
    For strings, is particularly nice, \emph{\texttt{in}} checks if a string
is contained in another one:

    \begin{Verbatim}[commandchars=\\\{\}]
{\color{incolor}In [{\color{incolor}60}]:} \PY{l+s}{\PYZsq{}}\PY{l+s}{cde}\PY{l+s}{\PYZsq{}} \PY{o+ow}{in} \PY{l+s}{\PYZsq{}}\PY{l+s}{abcdefgh}\PY{l+s}{\PYZsq{}}
\end{Verbatim}

            \begin{Verbatim}[commandchars=\\\{\}]
{\color{outcolor}Out[{\color{outcolor}60}]:} True
\end{Verbatim}
        
    Again, it's not a necessary distinction, but one way to distinguish
tuples and lists is that tuples are a collection of different things,
here a name, and x and y coordinates, whereas a list is a collection of
similar things, like if we wanted a list of those coordinates:

    \begin{Verbatim}[commandchars=\\\{\}]
{\color{incolor}In [{\color{incolor}61}]:} \PY{n}{positions} \PY{o}{=} \PY{p}{[}
                      \PY{p}{(}\PY{l+s}{\PYZsq{}}\PY{l+s}{Bob}\PY{l+s}{\PYZsq{}}\PY{p}{,}\PY{l+m+mf}{0.0}\PY{p}{,}\PY{l+m+mf}{21.0}\PY{p}{)}\PY{p}{,}
                      \PY{p}{(}\PY{l+s}{\PYZsq{}}\PY{l+s}{Cat}\PY{l+s}{\PYZsq{}}\PY{p}{,}\PY{l+m+mf}{2.5}\PY{p}{,}\PY{l+m+mf}{13.1}\PY{p}{)}\PY{p}{,}
                      \PY{p}{(}\PY{l+s}{\PYZsq{}}\PY{l+s}{Dog}\PY{l+s}{\PYZsq{}}\PY{p}{,}\PY{l+m+mf}{33.0}\PY{p}{,}\PY{l+m+mf}{1.2}\PY{p}{)}
                      \PY{p}{]}
\end{Verbatim}

    \subsection{Dictionaries}\label{dictionaries}

\begin{itemize}
\itemsep1pt\parskip0pt\parsep0pt
\item
  called ``mappings'' or ``associative arrays'' in other languages

  \begin{itemize}
  \itemsep1pt\parskip0pt\parsep0pt
  \item
    from a set of \emph{keys} to a set of \emph{values}
  \end{itemize}
\item
  You can define, modify, view, lookup, and delete the key-value pairs
  in the dictionary
\end{itemize}

    The index in a dictionary is called the \emph{key}, and the
corresponding dictionary entry is the \emph{value}. A dictionary can use
(almost) anything as the key. Whereas lists are formed with square
brackets {[}{]}, dictionaries use curly brackets \{\}:

    \begin{Verbatim}[commandchars=\\\{\}]
{\color{incolor}In [{\color{incolor}62}]:} \PY{n}{ages} \PY{o}{=} \PY{p}{\PYZob{}}\PY{l+s}{\PYZdq{}}\PY{l+s}{Rick}\PY{l+s}{\PYZdq{}}\PY{p}{:} \PY{l+m+mi}{46}\PY{p}{,} \PY{l+s}{\PYZdq{}}\PY{l+s}{Bob}\PY{l+s}{\PYZdq{}}\PY{p}{:} \PY{l+m+mi}{86}\PY{p}{,} \PY{l+s}{\PYZdq{}}\PY{l+s}{Fred}\PY{l+s}{\PYZdq{}}\PY{p}{:} \PY{l+m+mi}{21}\PY{p}{\PYZcb{}}
         \PY{k}{print}\PY{p}{(}\PY{l+s}{\PYZdq{}}\PY{l+s}{Rick}\PY{l+s}{\PYZsq{}}\PY{l+s}{s age is }\PY{l+s}{\PYZdq{}}\PY{p}{,}\PY{n}{ages}\PY{p}{[}\PY{l+s}{\PYZdq{}}\PY{l+s}{Rick}\PY{l+s}{\PYZdq{}}\PY{p}{]}\PY{p}{)}
\end{Verbatim}

    \begin{Verbatim}[commandchars=\\\{\}]
Rick's age is  46
    \end{Verbatim}

    There's also a convenient way to create dictionaries without having to
quote the keys.

    \begin{Verbatim}[commandchars=\\\{\}]
{\color{incolor}In [{\color{incolor}63}]:} \PY{n+nb}{dict}\PY{p}{(}\PY{n}{Rick}\PY{o}{=}\PY{l+m+mi}{46}\PY{p}{,}\PY{n}{Bob}\PY{o}{=}\PY{l+m+mi}{86}\PY{p}{,}\PY{n}{Fred}\PY{o}{=}\PY{l+m+mi}{20}\PY{p}{)}
\end{Verbatim}

            \begin{Verbatim}[commandchars=\\\{\}]
{\color{outcolor}Out[{\color{outcolor}63}]:} \{'Bob': 86, 'Fred': 20, 'Rick': 46\}
\end{Verbatim}
        
    The \textbf{len()} command works on both tuples and dictionaries:

    \begin{Verbatim}[commandchars=\\\{\}]
{\color{incolor}In [{\color{incolor}64}]:} \PY{n+nb}{len}\PY{p}{(}\PY{n}{t}\PY{p}{)}
\end{Verbatim}

            \begin{Verbatim}[commandchars=\\\{\}]
{\color{outcolor}Out[{\color{outcolor}64}]:} 4
\end{Verbatim}
        
    \begin{Verbatim}[commandchars=\\\{\}]
{\color{incolor}In [{\color{incolor}65}]:} \PY{n+nb}{len}\PY{p}{(}\PY{n}{ages}\PY{p}{)}
\end{Verbatim}

            \begin{Verbatim}[commandchars=\\\{\}]
{\color{outcolor}Out[{\color{outcolor}65}]:} 3
\end{Verbatim}
        
    \section{Modules}\label{modules}

Are Python files (*.py) with functions, objects and/or variables.

    \textbf{why modules?}

\begin{itemize}
\item
  use classes and functions defined in another file
\item
  copy pasting code
\end{itemize}

    Like Java \textbf{\texttt{import}}, C++ \emph{include}

Three formats of the command:

\begin{verbatim}
    import somefile (as sf)
    
    from somefile import *
    
    from somefile import className
    
\end{verbatim}

    \paragraph{\texttt{import somefile}}\label{import-somefile}

\begin{itemize}
\item
  Everything in \texttt{somefile.py} gets imported
\item
  to refer to something in the file, append the text \texttt{somefile.}
  to the front of its name:

  \begin{itemize}
  \itemsep1pt\parskip0pt\parsep0pt
  \item
    \texttt{somefile.myFunction(34)}
  \end{itemize}
\end{itemize}

    \paragraph{\texttt{import somefile as sf}}\label{import-somefile-as-sf}

\begin{itemize}
\item
  import with an alias
\item
  to refer to something in the file, append the text \texttt{sf.} to the
  front of its name:

  \begin{itemize}
  \itemsep1pt\parskip0pt\parsep0pt
  \item
    \texttt{sf.myFunction(34)}
  \end{itemize}
\end{itemize}

    \paragraph{\texttt{from somefile import *}}\label{from-somefile-import}

\begin{itemize}
\item
  everything in \texttt{somefile.py} gets imported
\item
  to refer to anything in the module, just use its name

  \begin{itemize}
  \itemsep1pt\parskip0pt\parsep0pt
  \item
    everything in the module is now in the current namespace
  \end{itemize}
\item
  CAUTION! using this import command can easily overwrite the definition
  of an existing function or variable
\end{itemize}

    \paragraph{\texttt{from somefile import myFunction}**}\label{from-somefile-import-myfunction}

\begin{itemize}
\item
  Only the item \texttt{myFunction} in \texttt{somefile.py} gets
  imported
\item
  you can just use it without a module prefix

  \begin{itemize}
  \itemsep1pt\parskip0pt\parsep0pt
  \item
    It's brought into the current namespace
  \end{itemize}
\item
  CAUTION! This will overwrite the definition of this particular name if
  it is already defined in the current namespace!
\end{itemize}

    \subsubsection{Commonly Used Modules}\label{commonly-used-modules}

Some useful modules to import coming along with Python installation:

\begin{itemize}
\item
  \textbf{\texttt{sys}} - Lots of handy stuff

  \begin{itemize}
  \item
    \texttt{argv}
  \item
    \texttt{sys.path.append}
  \end{itemize}
\item
  \textbf{\texttt{os}} - OS specific code

  \begin{itemize}
  \item
    \texttt{listdir, system, \ldots{}}
  \item
    \texttt{os.path} - directory processing
  \end{itemize}
\item
  \textbf{\texttt{math}} - mathematical code

  \begin{itemize}
  \itemsep1pt\parskip0pt\parsep0pt
  \item
    \texttt{sin, cos, exp, log, sqrt, \ldots{}}
  \end{itemize}
\item
  \textbf{\texttt{random}} - random number code

  \begin{itemize}
  \itemsep1pt\parskip0pt\parsep0pt
  \item
    \texttt{uniform, choice, shuffle}
  \end{itemize}
\item
  \textbf{\texttt{argparse}} - script input parsing
\end{itemize}

    \paragraph{For scientists}\label{for-scientists}

\begin{itemize}
\item
  \textbf{\texttt{numpy}} - basis for numerics

  \begin{itemize}
  \item
    powerful \texttt{numpy} array
  \item
    efficient functions on \texttt{numpy} arrays
  \end{itemize}
\item
  \textbf{\texttt{scipy}}` - advanced methods

  \begin{itemize}
  \item
    linear algebra
  \item
    integration, ODE solving, \ldots{}
  \end{itemize}
\item
  \textbf{\texttt{matplotlib}} - matlab like plotting

  \begin{itemize}
  \itemsep1pt\parskip0pt\parsep0pt
  \item
    plotting
  \end{itemize}
\end{itemize}

    \textbf{Hands on!}

Import the \textbf{\texttt{time}} module and get the local time

    Tim Peters, one of the earliest and most prolific Python contributors,
wrote the ``Zen of Python'', which can be accessed via the ``import
this'' command \ldots{}

No matter how experienced a programmer you are, these are words to
meditate on.

    \section{References}\label{references}

\begin{itemize}
\itemsep1pt\parskip0pt\parsep0pt
\item
  \href{http://docs.python.org/2/tutorial/}{Python Tutorial}
\item
  \href{http://learnpythonthehardway.org/book/}{Learn Python the Hard
  Way}
\item
  \href{http://docs.python.org/2.7/}{Python Documentation}
\item
  \href{http://docs.python.org/2.7/reference/index.html}{Python Language
  Reference}
\end{itemize}

\subsection{Packages for Scientists}\label{packages-for-scientists}

Important libraries

\begin{itemize}
\itemsep1pt\parskip0pt\parsep0pt
\item
  \href{http://www.numpy.org}{Numpy}, the core numerical extensions for
  linear algebra and multidimensional arrays;
\item
  \href{http://www.scipy.org}{Scipy}, additional libraries for
  scientific programming;
\item
  \href{http://matplotlib.sf.net}{Matplotlib}, excellent plotting and
  graphing libraries;
\item
  \href{http://ipython.org}{IPython}, with the additional libraries
  required for the notebook interface.
\item
  \href{http://sympy.org}{Sympy}, symbolic math in Python
\item
  \href{http://pandas.pydata.org/}{Pandas} library for big data in
  Python
\end{itemize}

\subsection{Badass IPython Notebooks}\label{badass-ipython-notebooks}

\begin{itemize}
\itemsep1pt\parskip0pt\parsep0pt
\item
  Rob Johansson's \href{http://jrjohansson.github.io/}{excellent
  notebooks}, including
  \href{https://github.com/jrjohansson/scientific-python-lectures}{Scientific
  Computing with Python} and
  \href{https://github.com/jrjohansson/qutip-lectures}{Computational
  Quantum Physics with QuTiP} lectures;
\item
  \href{http://nbviewer.ipython.org/url/jakevdp.github.com/downloads/notebooks/XKCD_plots.ipynb}{XKCD
  style graphs in matplotlib};
\item
  \href{https://github.com/ipython/ipython/tree/master/examples/notebooks\#a-collection-of-notebooks-for-using-ipython-effectively}{A
  collection of Notebooks for using IPython effectively}
\item
  \href{https://github.com/ipython/ipython/wiki/A-gallery-of-interesting-IPython-Notebooks}{A
  gallery of interesting IPython Notebooks}
\item
  \href{https://github.com/invisibleroads/crosscompute-tutorials}{Cross-disciplinary
  computational analysis IPython Notebooks From Hadoop World 2012}
\item
  \href{http://nbviewer.ipython.org/urls/raw.github.com/tbekolay/pyconca2012/master/QuantitiesTutorial.ipynb}{Quantites}
  Units in Python.

  \begin{itemize}
  \itemsep1pt\parskip0pt\parsep0pt
  \item
    \href{http://www.southampton.ac.uk/~fangohr/blog/}{Another units
    module is here}
  \end{itemize}
\end{itemize}


    % Add a bibliography block to the postdoc
    
    
    
    \end{document}
